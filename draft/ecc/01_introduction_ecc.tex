\section{Introduction}

Many chemical and petrochemical processes involve states that evolve over both space and time. These processes are modeled as distributed parameter systems (DPSs) using partial differential equations (PDEs) \cite{ray1981advanced}.
The infinite-dimensional nature of DPSs poses distinct challenges for control and estimation. Two primary strategies exist for addressing this: Early Lumping and Late Lumping.
Early Lumping discretizes the system early in the modeling process to enable standard control techniques \cite{davison1976robust}, but this can introduce inaccuracies due to model reduction errors \cite{moghadam2012infinite}.
In contrast, Late Lumping retains the system’s infinite-dimensional structure until the final controller implementation stage, offering higher fidelity at the cost of greater complexity.

Several Late Lumping approaches have been employed to control convection-reaction and diffusion-convection-reaction systems modeled by hyperbolic and parabolic PDEs, respectively.
Robust and boundary feedback control of plug flow reactors have been demonstrated in \cite{christofides1996feedback, krstic2008backstepping}, while \cite{xu2016state} addresses state feedback design for countercurrent heat exchangers.
The role of dispersion in axial dispersion tubular reactors is considered in \cite{christofides1998robust}, and low-dimensional predictive controllers based on modal decomposition have been explored in \cite{dubljevic2006predictive2}.
A comprehensive observer-based MPC strategy is proposed in \cite{khatibi2021model} for axial dispersion tubular reactors with recycle, combining diffusion, convection, and feedback under input constraints.
State reconstruction for DPSs has also been addressed using discrete-time Luenberger observers without spatial discretization, a key feature consistent with the late lumping paradigm \cite{dochain2000state, dochain2001state, alonso2004optimal, ali2015review}.

Delay systems, another class of infinite-dimensional systems, are often represented either as delay differential equations or transport PDEs, with the latter offering advantages in systems with spatial dynamics \cite{krstic2009book}.
While input/output delays have been widely studied in chemical engineering using cascade PDE models \cite{Hiratsuka1969IEEE, mohammadi2012lq, Guilherme2019ACC}, state delays are less commonly addressed.
Notable exceptions include heat exchanger systems with stream passage delays \cite{ozorio2019heat} and plug flow reactors with recycle delays that omit dispersion \cite{qi2021output}.
Regarding diffusion-convection-reaction systems, even the work of \cite{khatibi2021model}—which remains one of the most complete applications of Late Lumping for distributed chemical reactors—assumes an instantaneous recycle stream, leaving a gap in the literature concerning systems where recycle imposes a state delay.

This work addresses an axial dispersion tubular reactor with recycle, modeled as a coupled system of parabolic and hyperbolic PDEs to capture both spatial dispersion and recycle-induced state delay.
A closed-form resolvent operator is derived to preserve the system’s infinite-dimensional structure using the Late Lumping approach.
The system is then discretized using the Cayley-Tustin method, a Crank-Nicolson-type scheme that conserves the dynamics without requiring model reduction \cite{havu2007cayley, xu2017linear}.
A discrete-time infinite-dimensional Luenberger observer is designed to reconstruct unmeasured states, enabling output feedback MPC.
Simulations show that the proposed controller successfully stabilizes the otherwise unstable system under input constraints.