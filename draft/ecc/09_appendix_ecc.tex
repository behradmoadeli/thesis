\appendix

\subsection{Resolvent Operator Derivation} \label{app:resolvent}
To obtain the closed form of the resolvent operator $\mathfrak{R}(s, \mathfrak{A}) = (sI-\mathfrak{A})^{-1}$, it can be treated as an operator that maps the initial condition or the input to the Laplace transform of the state of the system $\underline{X}(\zeta, s)$. In \eqref{eq:resolvent}, Laplace transform is applied to the LTI representation of the system for both zero-input response and zero-state response to obtain a general expression for the resolvent operator. The goal is to obtain the solution for $\underline{X}(\zeta, s)$ and compare it with the general expression obtained in \eqref{eq:resolvent} to get the closed form expression for the resolvent operator. First step is to apply Laplace transform to the original system of PDEs in \eqref{eq:operator_A}.

\begin{equation} \label{eq:resolvent}
    \begin{aligned}
        \dot{\underline{x}}(\zeta, t) &= \mathfrak{A} \underline{x}(\zeta, t) + \mathfrak{B} u(t) \xrightarrow{\mathcal{L}}\\
        s \underline{X}(\zeta,s) - \underline{x}(\zeta,0) &= \mathfrak{A} \underline{X}(\zeta,s) + \mathfrak{B} U(s)\\
        &\hspace{-7.5em
        }\begin{cases}
            \xrightarrow{u = 0} &\underline{X}(\zeta,s) = (sI - \mathfrak{A})^{-1} \underline{x}(\zeta,0) = \mathfrak{R}(s, \mathfrak{A}) \underline{x}(\zeta,0)\\
            \xrightarrow{\underline{x}(0, \zeta)}& \underline{X}(\zeta,s) = (sI - \mathfrak{A})^{-1} \mathfrak{B} U(s) = \mathfrak{R}(s, \mathfrak{A}) \mathfrak{B} U(s)
        \end{cases}
    \end{aligned}
\end{equation}

The second order derivative term is decomposed to two first order PDEs, constructing a new $3 \times 3$ system of first order ODEs with respect to $\zeta$ after Laplace transformation, as shown in \eqref{eq:laplace_transformed} along with the solution.

\begin{equation} \label{eq:laplace_transformed}
    \begin{aligned}
        \partial_\zeta \overbrace{\begin{bmatrix}
            X_1(\zeta,s)\\ \partial_\zeta X_1(\zeta,s)\\ X_2(\zeta,s)
        \end{bmatrix}}^{\underline{\tilde{X}}(\zeta,s)} &= \overbrace{\begin{bmatrix}
            0 & 1 & 0\\
            \frac{s-k}{D} & \frac{v}{D} & 0\\
            0 & 0 & s\tau
            \end{bmatrix}}^{P(s)} \, \begin{bmatrix}
                X_1(\zeta,s)\\ \partial_\zeta X_1(\zeta,s)\\ X_2(\zeta,s)
            \end{bmatrix} \\
            &+ \underbrace{\begin{bmatrix}
                0\\ -\frac{x_1(\zeta,0)}{D} + v(1-R) \delta(\zeta) U(s)\\ -\tau x_2(\zeta,0)
            \end{bmatrix}}_{Z(\zeta,s)} \\
            % \partial_\zeta \underline{\tilde{X}}(\zeta,s) &= P(s) \underline{\tilde{X}}(\zeta,s) + Z(\zeta,s) \\
            \Rightarrow \underline{\tilde{X}}(\zeta,s) &= \underbrace{e^{P(s)\zeta}}_{T(\zeta,s)} \underline{\tilde{X}}(0,s) + \int_0^\zeta \underbrace{e^{P(s)(\zeta - \eta)}}_{F(\zeta, \eta)} Z(\eta,s) d\eta
    \end{aligned}
\end{equation}

Since the boundary conditions are not homogeneous, $\underline{\tilde{X}}(0,s)$ needs to be obtained by solving the system of algebraic equations given in \eqref{eq:BC_AE}.

\begin{equation} \label{eq:BC_AE}
\begin{aligned}
        &\overbrace{\begin{bmatrix}
            -v & D & Rv\\
            T_{11}(1,s) & T_{12}(1,s) & -T_{33}(1,s)\\
            T_{21}(1,s) & T_{22}(1,s) & 0
        \end{bmatrix}}^{M^{-1}(s)} \underline{\tilde{X}}(0,s) =\\ 
        &\underbrace{\int_0^1 \begin{bmatrix}
            0\\ F_{33}(1, \eta) Z_3(\eta,s) - F_{12}(1, \eta) Z_2(\eta,s)\\ -F_{22}(1, \eta) Z_2(\eta,s)
        \end{bmatrix} d\eta}_{\underline{b}(s)} \\
        % \Rightarrow &\underline{\tilde{X}}(0,s) = M(s) \underline{b}(s)
\end{aligned}
\end{equation}

Having access to $\underline{\tilde{X}}(0,s)$, the resolvent operator can be explicitly derived as shown in \eqref{eq:resolvent_u} and \eqref{eq:resolvent_x} for zero-state and zero-input cases, respectively.


\begin{equation} \label{eq:resolvent_u}
    \begin{aligned}
        &\underline{x}(\zeta,0) = 0 \Rightarrow \mathfrak{R}(s, \mathfrak{A}) \mathfrak{B} (\cdot) = \begin{bmatrix}
            \mathfrak{R}_{1} \mathfrak{B}\\
            \mathfrak{R}_{2} \mathfrak{B}
        \end{bmatrix} (\cdot) \Rightarrow\\
        &\mathfrak{R}_{1} \mathfrak{B} = -v(1-R) \bigl[ \sum_{j=1}^{2} T_{1j}(\zeta) (M_{j2} T_{12}(1) + M_{j3} T_{22}(1)) \\
        &\hspace{3em} - T_{12}(\zeta) \bigr] (\cdot)\\
        &\mathfrak{R}_{2} \mathfrak{B} = -v(1-R) \left[ T_{33}(\zeta) (M_{32} T_{12}(1) + M_{33} T_{22}(1)) \right] (\cdot)
    \end{aligned}
\end{equation}

\begin{equation} \label{eq:resolvent_x}
    \begin{aligned}
        &U(s) = 0 \Rightarrow \mathfrak{R}(s, \mathfrak{A}) \underline{(\cdot)} = \begin{bmatrix}
            \mathfrak{R}_{11} & \mathfrak{R}_{12}\\
            \mathfrak{R}_{21} & \mathfrak{R}_{22}
        \end{bmatrix} \begin{bmatrix}
            (\cdot)_1\\ (\cdot)_2
        \end{bmatrix} \Rightarrow\\
        &\mathfrak{R}_{11} = \sum_{j=1}^2 \frac{T_{1j}(\zeta)}{D} \int_0^1 \left[ M_{j2} F_{12}(1,\eta) + M_{j3} F_{22}(1,\eta) \right] (\cdot)_1 d\eta\\
        &\hspace{2.5em} -\frac{1}{D} \int_0^{\zeta} F_{12}(\zeta, \eta) (\cdot)_1 d\eta\\
        &\mathfrak{R}_{12} = \sum_{j=1}^2 -\tau T_{1j}(\zeta) \int_0^1 M_{j2} F_{33}(1,\eta) (\cdot)_2 d\eta\\
        &\mathfrak{R}_{21} = \frac{T_{33}(\zeta)}{D} \int_0^1 \left[ M_{32} F_{12}(1,\eta) + M_{33} F_{22}(1,\eta) \right] (\cdot)_1 d\eta\\
        &\mathfrak{R}_{22} = -\tau T_{33}(\zeta) \int_0^1 M_{32} F_{33}(1,\eta) (\cdot)_2 d\eta\\
        &\hspace{2.5em} -\tau \int_0^{\zeta} F_{33}(\zeta, \eta) (\cdot)_2 d\eta \\
    \end{aligned}
\end{equation}

\subsection{Standard QP Representation for the MPC Optimization Problem} \label{app:QP}
Simple algebraic manipulation can be performed to express future state trajectories in terms of the current estimated state and a sequence of future inputs, allowing the optimization problem in \eqref{eq:MPC_finite_time} to be reformulated into the standard quadratic programming (QP) structure shown in \eqref{eq:MPC_QP}. This reformulation enables the use of conventional QP solvers for efficient real-time implementation.

\begin{equation} \label{eq:MPC_QP}
    \begin{aligned}
        \min_{U} &J = U^\top \langle I,H \rangle U + 2U^\top \langle I, P \underline{\hat{x}}(\zeta, k|k) \rangle \\
        \text{s.t.} &\qquad U^{min} \leq U \leq U^{max} \\
        &\qquad T_u \underline{\hat{x}}(\zeta, k|k) + S_u U = 0
        \, \\
        \text{with } &H = \\
        &\hspace{-3.5em }\begin{bmatrix}
            \mathfrak{B}_d^* \mathfrak{P} \mathfrak{B}_d + \mathfrak{F} & \mathfrak{B}_d^* \mathfrak{A}_d^* \mathfrak{P} \mathfrak{B}_d & \cdots &  \mathfrak{B}_d^* {\mathfrak{A}_d^*}^{N-1} \mathfrak{P} \mathfrak{B}_d \\
            \mathfrak{B}_d^* \mathfrak{P} \mathfrak{A}_d \mathfrak{B}_d & \mathfrak{B}_d^* \mathfrak{P} \mathfrak{B}_d + \mathfrak{F} & \cdots & \mathfrak{B}_d^* {\mathfrak{A}_d^*}^{N-2} \mathfrak{P} \mathfrak{B}_d \\
            \vdots & \vdots & \ddots & \vdots \\
            \mathfrak{B}_d^* \mathfrak{P} {\mathfrak{A}_d}^{N-1} \mathfrak{B}_d & \mathfrak{B}_d^* \mathfrak{P} {\mathfrak{A}_d}^{N-2} \mathfrak{B}_d & \cdots & \mathfrak{B}_d^* \mathfrak{P} \mathfrak{B}_d + \mathfrak{F}
        \end{bmatrix} \\
        P = &\begin{bmatrix}
            \mathfrak{B}_d^* \mathfrak{P} {\mathfrak{A}_d} &
            \mathfrak{B}_d^* \mathfrak{P} {\mathfrak{A}_d}^{2}  &
            \hdots &
            \mathfrak{B}_d^* \mathfrak{P} {\mathfrak{A}_d}^{N} 
        \end{bmatrix}^\top \\
        T_u (\cdot) = &\begin{bmatrix}
            \langle {\mathfrak{A}_d}^{N} (\cdot), \underline{\phi_u} \rangle
        \end{bmatrix} \\
        S_u = &\begin{bmatrix}
            \langle {\mathfrak{A}_d}^{N-1} \mathfrak{B}_d, \underline{\phi_u} \rangle & 
            \langle {\mathfrak{A}_d}^{N-2} \mathfrak{B}_d, \underline{\phi_u} \rangle &
            \hdots &
            \langle \mathfrak{B}_d, \underline{\phi_u} \rangle
        \end{bmatrix} \\
        U = &\begin{bmatrix}
            u(k+1|k) & u(k+2|k) & \hdots & u(k+N|k)
        \end{bmatrix}^\top
    \end{aligned}
\end{equation}