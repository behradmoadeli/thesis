\section{Methodology}

\subsection{Model representation}

The chemical process depicted in Fig.~\ref{fig:reactor_scheme} illustrates a chemical reaction within an axial dispersion tubular reactor \cite{levenspiel1998chemical} where reactant $A$ is converted into products. The reactor features a recycle mechanism, allowing a portion of the product stream to re-enter the reactor, ensuring the consumption of any unreacted substrate. The dynamics of the reactant concentration can be described by the second-order parabolic PDE given by \eqref{eq:PDE_original_model}

% , a common class of equations used to characterize diffusion-convection-reaction systems \cite{jensen1982bifurcation}. The resulting PDE that describes the reactor model is , subject to the boundary conditions in \eqref{eq:BC}, obtained by utilizing first-principle modeling through relevant mass balance relations on an infinitesimally thin disk element along the longitudinal axis of the reactor.

\begin{figure}[!htbp] 
    \centering
    \begin{tikzpicture}
        \node (pfr) [cylinder, draw, minimum height=3cm, minimum width=1cm, shape aspect=1, shape border rotate=180, cylinder uses custom fill, cylinder end fill=gray!45, cylinder body fill=gray!15] {\hspace{-0.2cm}$A \rightarrow Products$};
        \node (pfr_inlet) [circle, left of=pfr, xshift=-0.5cm, fill=black, draw, inner sep=0pt, minimum size=0.25cm, scale=0.5] {};
        \node (pfr_outlet) [circle, at={(pfr.east)}, shift={(-0.25cm,0)}, fill=black, draw, inner sep=0pt, minimum size=0.25cm, scale=0.5] {};
        \node (recycle_right) [circle, right of=pfr_outlet, fill=black, draw, inner sep=0pt, minimum size=0.25cm, scale=0.5] {};
        \node (recycle_left) [circle, left of=pfr_inlet, fill=black, draw, inner sep=0pt, minimum size=0.25cm, scale=0.5] {};
        
        \draw[dotted, thick] ([yshift=0.5cm]pfr_inlet.center) -- node[at end, below, yshift=0.1cm] {$\zeta = 0$} ([yshift=-0.65cm]pfr_inlet.center);
        \draw[dotted, thick] ([yshift=0.5cm]pfr_outlet.center) -- node[at end, below, yshift=0.1cm] {$\zeta = 1$} ([yshift=-0.65cm]pfr_outlet.center);
        
        \node[below of=recycle_left, node distance=1.3cm, anchor=north west, xshift=-0.2cm] {$R \, C_A(1, t-\tau)$};
        \node[above of=pfr_inlet, node distance=0.75cm,] {$C_A(0, t)$};
        \node[above of=pfr_outlet, node distance=0.75cm,] {$C_A(1, t)$};
        
        \draw [arrow_2] (pfr_outlet) -- node[near end, above] {$y(t)$} ++(2,0);
        \draw [arrow_2] (pfr_inlet) ++(-2,0) coordinate(start) -- node[near start, above] {$C_A^{feed}(t)$} (pfr_inlet);
        \draw [arrow_2] (recycle_right) -- ++(0,-1.25) -| (recycle_left);
        
    \end{tikzpicture}
    \caption{Axial tubular reactor with recycle stream.}
    \label{fig:reactor_scheme}
    % \pdfbookmark[2]{Figure: Reactor Scheme}{fig:reactor_scheme}
\end{figure}

\begin{equation} \label{eq:PDE_original_model}
    \dot{C}_A(\zeta, t) = D \partial_{\zeta \zeta} C_A(\zeta, t) - v \partial_\zeta C_A(\zeta, t) - r (C_A)
\end{equation}


Here, $C_A(\zeta, t)$ is the concentration of reactant $A$ along the reactor. The physical parameters $D$ and $v$ represent the diffusion coefficient and flow velocity along the reactor, respectively. Physical parameters are assumed to be constant, hence changes in temperature or pressure will not affect the reactor model. The coordinate system in space and time is represented by $\zeta$ and $t$, where $\zeta \in [0, 1]$ and $t \in [0, \infty)$. In addition, $r(C_A)$ is the reaction rate of the reactant in general, which is often a non-linear function of $C_A$. Therefore, the model is further linearized around its steady-state, followed by introducing the deviation variable $c(\zeta, t) = C_A(\zeta, t) - C_{A, ss}(\zeta)$, where $C_{A, ss}(\zeta)$ is the steady-state concentration of the reactant. The linearized model is then given by \eqref{eq:PDE_linearized_model}.

\begin{equation} \label{eq:PDE_linearized_model}
    \dot{c}(\zeta, t) = D \partial_{\zeta \zeta} c(\zeta, t) - v \partial_\zeta c(\zeta, t) - k_r c(\zeta, t)
\end{equation}

Here, $k_r \equiv \left. \dfrac{\partial r(C_A)}{\partial C_A} \right|_{C_{A, ss}}$ is the linearized reaction rate coefficient in the vicinity of the steady-state. The system input is defined as $u(t) \equiv C_{A}^{feed} - C_{A, ss}^{feed}$, representing the deviation of the concentration of the reactant being fed into the reactor from its steady-state value. The output of the system is also considered as the deviation of the concentration of the reactant being measured at the reactor outlet from its steady-state value, denoted as $y(t)$. 

To accurately represent the behavior of the given axial dispersion tubular reactor, Dankwerts boundary conditions are applied; as they effectively capture deviations from ideal mixing and piston flow while assuming negligible transport lags in connecting lines \cite{danckwerts1993continuous}.
The inlet boundary condition is modified to reflect the mixing of the input stream with the delayed state, i.e. the recycled reactant concentration coming from the reactor outlet, occurring $\tau$ seconds earlier.
These boundary conditions are therefore summarized in \eqref{eq:BC}, with $R$ and $\tau$ denoting the recycle ratio and the residence time in the recycle stream, respectively.

\begin{align} \label{eq:BC}
    \begin{cases}
        &D \partial_\zeta c(0, t) - v c(0, t) = -v \left[ R c(1, t-\tau) + (1-R) u(t) \right] \\
        &\partial_\zeta c(1, t) = 0 \\
        &y(t) = c(1, t)
    \end{cases}
\end{align}

In the case where the problem involves similar forms of PDEs, an effective general practice to address delays in systems is to reformulate the problem such that the notion of delay is replaced with an alternative transport PDE. Therefore, a new state variable $\underline{x}(\zeta, t) \equiv [x_1(\zeta, t), x_2(\zeta, t)]^T$ is defined as a vector of functions, where $x_1(\zeta, t)$ represents the concentration within the reactor—analogous to $c(\zeta,t)$—and $x_2(\zeta, t)$ is introduced as a new state variable to account for the concentration along the recycle stream. The delay is thus modeled as a pure transport process, wherein the first state $x_1(\zeta, t)$ is transported from the reactor outlet to the inlet, experiencing a delay of $\tau$ time units while in the recycle stream. This makes all state variables expressed explicitly at a specific time instance $t$, resulting in the standard state-space form for a given infinite-dimensional linear time-invariant (LTI) system $\dot{\underline{x}} = \mathfrak{A} \underline{x} + \mathfrak{B} u$. Here, $\mathfrak{A}$ is a linear operator $\mathcal{L}(X)$ acting on a Hilbert space $X: L^2[0,1] \times L^2[0,1]$ as shown in \eqref{eq:operator_A}. Also, $\mathfrak{B}$ is a linear operator that maps the scalar input from input-space onto the state space, as defined in \eqref{eq:operator_B}.

\begin{equation} \label{eq:operator_A}
    \begin{aligned}
        \mathfrak{A} \equiv&
        \begin{bmatrix}
            D \partial_{\zeta \zeta} - v \partial_\zeta - k_r & 0 \\
            0 & \frac{1}{\tau} \partial_\zeta
        \end{bmatrix}\\
        D(\mathfrak{A}) =& \Bigl\{ \underline{x}(\zeta) = [x_1(\zeta), x_2(\zeta)]^T \in X:\\
        &\underline{x}(\zeta), \partial_\zeta \underline{x}(\zeta), \partial_{\zeta \zeta} \underline{x}(\zeta) \quad \mathrm{a.c.},\\
        &D \partial_\zeta x_1(0) - v x_1(0) = -v R x_2(0),\\
        &\partial_\zeta x_1(1) = 0,
        x_1(1) = x_2(1) \Bigr\}
    \end{aligned}
\end{equation}

\begin{equation} \label{eq:operator_B}
    \begin{aligned}
        \mathfrak{B} \equiv
        \begin{bmatrix}
            \delta(\zeta) \\
            0
        \end{bmatrix} v (1-R), \quad
        D(\mathfrak{B}) = \Bigl\{ u \in \mathbb{R} \Bigr\}
    \end{aligned}
\end{equation}

where $\delta(\zeta)$ is dirac delta function. This will enable the derivation of the system's spectrum using the eigenvalue problem. The characteristics equation of the system is obtained by solving the equation $det(\mathfrak{A}-\lambda_i~I)~=~0$ for $\lambda_i$, where $\lambda_i \in \mathbb{C}$ is the $i^{\text{th}}$ eigenvalue of the system and $I$ is the identity operator. Attempts to analytically solve this equation have failed; therefore, it is solved numerically using the parameters in Table~I. These parameters are carefully chosen to reflect key characteristics of the system, i.e. diffusion, convection, reaction, and delayed recycle. A negative reaction coefficient ($k_r$) is used to induce instability for analysis, a condition uncommon for isothermal reactors but possible in specific cases like autocatalytic or inhibitory reactions. Figure~2 depicts the resulting eigenvalue distribution in the complex plane, confirming instability of the linearized model near its steady state.


\begin{figure}[!htbp]
    \centering
    \includesvg[inkscapelatex=false, width=0.35\textwidth, keepaspectratio]{Figures/eig_val_dist_R_0.3.svg}
    \caption{Eigenvalues of operator $\mathfrak{A}$.}
    \label{fig:eigval_dist}
\end{figure}


\begin{table}[ht]
    \centering
    \caption{Physical Parameters for the System}
    \label{tab:pars}
    \begin{tabular}{|c|c|c|c|}
    \hline
    \textbf{Parameter}        & \textbf{Symbol} & \textbf{Value}     & \textbf{Unit}    \\ \hline
    Diffusivity               & $D$             & $2\times10^{-5}$   & ${m^2}/{s}$      \\ \hline
    Velocity                  & $v$             & $0.01$   & ${m}/{s}$        \\ \hline
    Reaction Constant         & $k_r$           & $-1.5$             & $s^{-1}$         \\ \hline
    Recycle Residence Time    & $\tau$          & $80$               & $s$              \\ \hline
    Recycle Ratio             & $R$             & $0.3$              & $-$              \\ \hline
    \end{tabular}
\end{table}

\subsection{Adjoint system}

Next step is to obtain the adjoint system operators $\mathfrak{A}^*$ and $\mathfrak{B}^*$. Utilizing the relation $\langle \mathfrak{A} \underline{x} + \mathfrak{B} u, \underline{y}\rangle = \langle \underline{x}, \mathfrak{A}^* \underline{y}\rangle + \langle u, \mathfrak{B}^* \underline{y}\rangle$, the adjoint operators $\mathfrak{A}^*$ and $\mathfrak{B}^*$ are obtained as shown in \eqref{eq:adjoint_A} and \eqref{eq:adjoint_B}, respectively.


\begin{equation} \label{eq:adjoint_A}
    \begin{aligned}
        {\mathfrak{A}}^{*} =&
        \begin{bmatrix}
            D \partial_{\zeta \zeta} + v \partial_\zeta - k_r & 0\\
            0 & -\frac{1}{\tau} \partial_\zeta
        \end{bmatrix}\\
        D(\mathfrak{A}^*) =& \Bigl\{ \underline{y} = [y_1, y_2]^T \in Y:\\
        &\underline{y}(\zeta), \partial_\zeta \underline{y}(\zeta), \partial_{\zeta \zeta} \underline{y}(\zeta) \quad \mathrm{a.c.},\\
        &D \partial_\zeta y_1(1) + v y_1(1) = \frac{1}{\tau} y_2(1), \\
        &R v y_1(0) = \frac{1}{\tau} y_2(0), 
        \partial_\zeta y_1(0) = 0 \Bigr\}
    \end{aligned}
\end{equation}

\begin{equation} \label{eq:adjoint_B}
    \mathfrak{B}^* (\cdot) = \Bigl[ v(1-R) \int_0^1 \delta(\zeta) (\cdot) d\zeta \quad , \quad 0 \Bigr]
\end{equation}

Once the adjoint operators are determined, the eigenfunctions $\{ \underline{\phi_i}(\zeta), \underline{\psi_i}(\zeta) \}$ (for $\mathfrak{A}$ and $\mathfrak{A}^*$, respectively) may be obtained and properly scaled following the calculation of eigenvalues. The set of scaled eigenfunctions will then form a bi-orthonormal basis for the Hilbert space $X$; which will be later used in the controller design. It is important to note that the system is not self adjoint, as the obtained adjoint operator and its domain are not the same as the original operator and its domain.

\subsection{Resolvent operator}

One must obtain the resolvent operator of the system $\mathfrak{R}(s, \mathfrak{A}) = (sI-\mathfrak{A})^{-1}$ prior to constructing the discrete-time representation of the system. One way to obtain it is by utilizing the modal characteristics of the system, resulting in an infinite-sum representation of the operator. While being a common practice in the literature, truncating the infinite-sum representation for numerical implementation may lead to a loss of accuracy. Another way to express the resolvent operator is by treating it as an operator that maps either the initial condition of the system $\underline{x}(\zeta,0)$ or the input $u(t)$, to the Laplace transform of the state of the system $\underline{X}(\zeta, s)$. This approach, although more computationally intensive, results in a closed form expression for the resolvent operator, preserving the infinite-dimensional nature of the system. In \eqref{eq:resolvent}, Laplace transform is applied to the LTI representation of the system for both zero-input response and zero-state response to obtain a general expression for the resolvent operator.

\begin{equation} \label{eq:resolvent}
    \begin{aligned}
        \dot{\underline{x}}(\zeta, t) &= \mathfrak{A} \underline{x}(\zeta, t) + \mathfrak{B} u(t) \xrightarrow{\mathcal{L}}\\
        s \underline{X}(\zeta,s) - \underline{x}(\zeta,0) &= \mathfrak{A} \underline{X}(\zeta,s) + \mathfrak{B} U(s)\\
        &\hspace{-7.5em}\begin{cases}
            \xrightarrow{u = 0} &\underline{X}(\zeta,s) = (sI - \mathfrak{A})^{-1} \underline{x}(\zeta,0) = \mathfrak{R}(s, \mathfrak{A}) \underline{x}(\zeta,0)\\
            \xrightarrow{\underline{x}(0, \zeta)}& \underline{X}(\zeta,s) = (sI - \mathfrak{A})^{-1} \mathfrak{B} U(s) = \mathfrak{R}(s, \mathfrak{A}) \mathfrak{B} U(s)
        \end{cases}
    \end{aligned}
    \end{equation}
    
    The goal is to obtain the solution for $\underline{X}(\zeta, s)$ and compare it with the general expression obtained in \eqref{eq:resolvent} to get the closed form expression for the resolvent operator. First step is to apply Laplace transform to the original system of PDEs in \eqref{eq:operator_A}. The second order derivative term is decomposed to two first order PDEs, constructing a new $3 \times 3$ system of first order ODEs with respect to $\zeta$ after Laplace transformation, as shown in \eqref{eq:laplace_transformed}.
    
    \begin{equation} \label{eq:laplace_transformed}
    \begin{aligned}
        \partial_\zeta \overbrace{\begin{bmatrix}
            X_1(\zeta,s)\\ \partial_\zeta X_1(\zeta,s)\\ X_2(\zeta,s)
        \end{bmatrix}}^{\underline{\tilde{X}}(\zeta,s)} &= \overbrace{\begin{bmatrix}
            0 & 1 & 0\\
            \frac{s+k_r}{D} & \frac{v}{D} & 0\\
            0 & 0 & s\tau
            \end{bmatrix}}^{P(s)} \, \begin{bmatrix}
                X_1(\zeta,s)\\ \partial_\zeta X_1(\zeta,s)\\ X_2(\zeta,s)
            \end{bmatrix} \\
            &+ \underbrace{\begin{bmatrix}
                0\\ -\frac{x_1(\zeta,0)}{D} + v(1-R) \delta(\zeta) U(s)\\ -\tau x_2(\zeta,0)
            \end{bmatrix}}_{Z(\zeta,s)} \\
            \Rightarrow \partial_\zeta \underline{\tilde{X}}(\zeta,s) &= P(s) \underline{\tilde{X}}(\zeta,s) + Z(\zeta,s)
    \end{aligned}
    \end{equation} 
    
    with solution given by \eqref{eq:ODE_solution}.
    
    \begin{equation} \label{eq:ODE_solution}
        \underline{\tilde{X}}(\zeta,s) = \underbrace{e^{P(s)\zeta}}_{T(\zeta,s)} \underline{\tilde{X}}(0,s) + \int_0^\zeta \underbrace{e^{P(s)(\zeta - \eta)}}_{F(\zeta, \eta)} Z(\eta,s) d\eta
    \end{equation}
    
    Since the boundary conditions are not homogeneous, $\underline{\tilde{X}}(0,s)$ needs to be obtained by solving the system of algebraic equations given in \eqref{eq:BC_AE}; which is the result of applying Danckwerts boundary conditions to the Laplace transformed system of PDEs at $\zeta = 1$.
    
    \begin{equation} \label{eq:BC_AE}
    \begin{aligned}
            &\overbrace{\begin{bmatrix}
                -v & D & Rv\\
                T_{11}(1,s) & T_{12}(1,s) & -T_{33}(1,s)\\
                T_{21}(1,s) & T_{22}(1,s) & 0
            \end{bmatrix}}^{M^{-1}(s)} \underline{\tilde{X}}(0,s) =\\ 
            &\underbrace{\int_0^1 \begin{bmatrix}
                0\\ F_{33}(1, \eta) Z_3(\eta,s) - F_{12}(1, \eta) Z_2(\eta,s)\\ -F_{22}(1, \eta) Z_2(\eta,s)
            \end{bmatrix} d\eta}_{\underline{b}(s)} \\
            \Rightarrow &\underline{\tilde{X}}(0,s) = M(s) \underline{b}(s)
    \end{aligned}
    \end{equation}
    
    Having access to $\underline{\tilde{X}}(0,s)$, the solution for $\underline{X}(\zeta,s)$ can be explicitly derived. The resolvent operator for zero-input and zero-state cases are therefore obtained in a closed form as shown in \eqref{eq:resolvent_x} and \eqref{eq:resolvent_u}, respectively.
    
    \begin{equation} \label{eq:resolvent_x}
    \begin{aligned}
        &U(s) = 0 \Rightarrow \mathfrak{R}(s, \mathfrak{A}) \underline{(\cdot)} = \begin{bmatrix}
            \mathfrak{R}_{11} & \mathfrak{R}_{12}\\
            \mathfrak{R}_{21} & \mathfrak{R}_{22}
        \end{bmatrix} \begin{bmatrix}
            (\cdot)_1\\ (\cdot)_2
        \end{bmatrix} \Rightarrow\\
        &\mathfrak{R}_{11} = \sum_{j=1}^2 \frac{T_{1j}(\zeta)}{D} \int_0^1 \left[ M_{j2} F_{12}(1,\eta) + M_{j3} F_{22}(1,\eta) \right] (\cdot)_1 d\eta\\
        &\hspace{2.5em} -\frac{1}{D} \int_0^{\zeta} F_{12}(\zeta, \eta) (\cdot)_1 d\eta\\
        &\mathfrak{R}_{12} = \sum_{j=1}^2 -\tau T_{1j}(\zeta) \int_0^1 M_{j2} F_{33}(1,\eta) (\cdot)_2 d\eta\\
        &\mathfrak{R}_{21} = \frac{T_{33}(\zeta)}{D} \int_0^1 \left[ M_{32} F_{12}(1,\eta) + M_{33} F_{22}(1,\eta) \right] (\cdot)_1 d\eta\\
        &\mathfrak{R}_{22} = -\tau T_{33}(\zeta) \int_0^1 M_{32} F_{33}(1,\eta) (\cdot)_2 d\eta\\
        &\hspace{2.5em} -\tau \int_0^{\zeta} F_{33}(\zeta, \eta) (\cdot)_2 d\eta
    \end{aligned}
    \end{equation}
    
    \begin{equation} \label{eq:resolvent_u}
    \begin{aligned}
        &\underline{x}(\zeta,0) = 0 \Rightarrow \mathfrak{R}(s, \mathfrak{A}) \mathfrak{B} (\cdot) = \begin{bmatrix}
            \mathfrak{R}_{1} \mathfrak{B}\\
            \mathfrak{R}_{2} \mathfrak{B}
        \end{bmatrix} (\cdot) \Rightarrow\\
        &\mathfrak{R}_{1} \mathfrak{B} = -v(1-R) \bigl[ \sum_{j=1}^{2} T_{1j}(\zeta) (M_{j2} T_{12}(1) + M_{j3} T_{22}(1)) \\
        &\hspace{3em} - T_{12}(\zeta) \bigr] (\cdot)\\
        &\mathfrak{R}_{2} \mathfrak{B} = -v(1-R) \left[ T_{33}(\zeta) (M_{32} T_{12}(1) + M_{33} T_{22}(1)) \right] (\cdot)
    \end{aligned}
    \end{equation}

Since the system generator $\mathfrak{A}$ is not self-adjoint, the resolvent operator for the adjoint system shall also be obtained. This is done in a similar manner as the original system, resulting in a closed-form expression for the adjoint resolvent operator $\mathfrak{R}^*(s, \mathfrak{A}^*)$. To avoid redundancy, the derivation of the resolvent operator for the adjoint system is not included in this manuscript.

\subsection{Cayley-Tustin time discretization}

To implement the system on digital controllers, it is necessary to transition to a discrete-time framework while preserving critical properties such as stability and controllability. The Cayley-Tustin time-discretization method achieves this by mapping the continuous-time system to the discrete domain \cite{havu2007cayley, xu2017linear}. This Crank-Nicolson type of discretization is also known as the lowest order symplectic integrator in Gauss quadrature-based Runge-Kutta methods \cite{hairer2006geometric}. Considering $\Delta t$ as the sampling time, and assuming a piecewise constant input within time intervals (zero-order hold), the discrete-time representation $\underline{x}(\zeta, k) = \mathfrak{A}_d \underline{x}(\zeta, k-1) + \mathfrak{B}_d u(k)$ is obtained, with discrete-time operators $\mathfrak{A}_d$ and $\mathfrak{B}_d$ defined in \eqref{eq:discrete_AB}, where $\alpha = 2/{\Delta t}$.

\begin{equation} \label{eq:discrete_AB}
    \begin{bmatrix}
        \mathfrak{A}_d & \mathfrak{B}_d
    \end{bmatrix} = 
    \begin{bmatrix}
        -I + 2\alpha \mathfrak{R}(\alpha, \mathfrak{A}) & \sqrt{2\alpha} \mathfrak{R}(\alpha, \mathfrak{A}) \mathfrak{B}
    \end{bmatrix}
\end{equation}

As required for systems with nonself-adjoint generators, the adjoint discrete-time operators $\mathfrak{A}_d^*$ and $\mathfrak{B}_d^*$ are also obtained in a similar manner, as shown in \eqref{eq:discrete_AB_star}.

\begin{equation} \label{eq:discrete_AB_star}
    \begin{bmatrix}
        \mathfrak{A}_d^* & \mathfrak{B}_d^*
    \end{bmatrix} = 
    \begin{bmatrix}
        -I + 2\alpha \mathfrak{R}^*(\alpha, \mathfrak{A}^*) & \sqrt{2\alpha} \mathfrak{B}^* \mathfrak{R}^*(\alpha, \mathfrak{A}^*)
    \end{bmatrix}
\end{equation}

\subsection{Model predictive control design}

The proposed MPC, as shown in Fig.~\ref{fig:block_diagram}, is developed in this section with the goal of stabilizing the given unstable infinite-dimensional system within an optimal framework while satisfying input constraints. An infinite-time open-loop objective function sets the foundation of the controller design in the discrete-time setting at each sampling instant $k$, which consists of a weighted sum of state deviations and actuation costs for all future time instances, subject to the system dynamics and input constraints, as shown in \eqref{eq:MPC_inf_time}.

\begin{figure}[!htbp]
    \centering
    \begin{tikzpicture}[node distance=2cm, scale=0.75, transform shape]
        \node (plant) [block] {Plant};
        \node (MPC) [block, below of=plant] {MPC};
        \draw [arrow] (plant.south) -- node[midway, right] {$\underline{x}(\zeta,k)$} (MPC.north);
        \draw [arrow] (MPC.west) -- ++(-1,0) |- node[near end, above] {$u(k)$} (plant.west);
        \draw [arrow] (plant.east) -- node[midway, above] {$y(k)$} ++(1,0);
    \end{tikzpicture}
    \caption{Proposed full-state feedback model predictive control system.}
    \label{fig:block_diagram}
\end{figure}

\begin{equation} \label{eq:MPC_inf_time}
    \begin{aligned}
        \min_{U} \quad \sum_{l=0}^{\infty} &\langle \underline{x}(\zeta, k+l | k), \mathfrak{Q} \underline{x}(\zeta, k+l | k) \rangle \\
        + &\langle u(k+l+1 | k), \mathfrak{F} u(k+l+1|k) \rangle \\
        \, \\
        \text{s.t.} \quad &\underline{x}(\zeta, k+l | k) = \mathfrak{A}_d \underline{x}(\zeta, k+l-1 | k) + \mathfrak{B}_d u(k+l | k) \\
        &u^{min} \leq u(k+l | k) \leq u^{max}
    \end{aligned}
\end{equation}

where $\mathfrak{Q}$ and $\mathfrak{F}$ are positive definite operators of appropriate dimensions, responsible for penalizing state deviations and actuation costs, respectively. The notation $(k+l|k)$ indicates the future time states or input instance $k+l$ obtained at time $k$. The infinite-time optimization problem may be reduced to a finite-time setup by assigning zero-input beyond a certain control horizon $N$, resulting in the optimization problem in \eqref{eq:MPC_finite_time}.

\begin{equation} \label{eq:MPC_finite_time}
    \begin{aligned}
        \min_{U} \quad \sum_{l=0}^{N-1} &\langle \underline{x}(\zeta, k+l | k), \mathfrak{Q} \underline{x}(\zeta, k+l | k) \rangle \\
        + &\langle u(k+l+1 | k), \mathfrak{F} u(k+l+1|k) \rangle \\
        + &\langle \underline{x}(\zeta, k+N | k), \mathfrak{P} \underline{x}(\zeta, k+N | k) \rangle \\
        \, \\
        \text{s.t.} \quad &\underline{x}(\zeta, k+l | k) = \mathfrak{A}_d \underline{x}(\zeta, k+l-1 | k) + \mathfrak{B}_d u(k+l | k) \\
        &u^{min} \leq u(k+l | k) \leq u^{max} \\
        & \langle \underline{x}(\zeta, k+N | k), \underline{\phi_u}(\zeta) \rangle = 0
    \end{aligned}
\end{equation}

Obtained as the solution to the discrete-time Lyapunov equation, $\mathfrak{P}$ is the terminal cost operator as shown in \eqref{eq:terminal_cost}; which can be proven to be positive definite only if the terminal state $\underline{x}(\zeta, k+N | k)$ is in a stable subspace. Therefore, an equality constraint is introduced to guarantee that the resulting quadratic optimization problem is convex. The terminal constraint is enforced by setting the projection of the terminal state onto the unstable subspace of the system to zero \cite{curtainbook, xu2017linear, khatibi2021model}. Here, $\underline{\phi_u}(\zeta)$ is the set of unstable eigenfunctions of the system, for all eigenvalues where $\operatorname{Re}(\lambda_u) \geq 0$.

\begin{equation} \label{eq:terminal_cost}
    \mathfrak{P} (\cdot) = \sum_{m=0}^{\infty} \sum_{n=0}^{\infty} 
    -\frac{
        \langle \underline{\phi_m} , \mathfrak{Q} \underline{\psi_n} \rangle
    }{
        \lambda_m + \overline{\lambda_n}
    }
    \langle (\cdot) , \underline{\psi_n} \rangle \phi_m
\end{equation}

One may further process the optimization problem in \eqref{eq:MPC_finite_time} to obtain a standard format for quadratic programming (QP) solvers by substituting the future states in terms of the current state and the sequence of future inputs using system dynamics expression. The resulting QP problem is given in \eqref{eq:MPC_QP}. The optimal input sequence $U$ is then obtained by solving the QP problem at each sampling instant $k$. To implement a receding horizon control strategy, only the first input of the optimal sequence $u(k+1|k)$ is applied to the system, and the optimization problem is solved again at the next sampling instant $k+1$.

\begin{equation} \label{eq:MPC_QP}
    \begin{aligned}
        \min_{U} &J = U^T \langle I,H \rangle U + 2U^T \langle I, P \underline{x}(\zeta, k|k) \rangle \\
        \text{s.t.} &\qquad U^{min} \leq U \leq U^{max} \\
        &\qquad T_u \underline{x}(\zeta, k|k) + S_u U = 0
        \, \\
        \text{with } &H = \\
        &\hspace{-3.5em }\begin{bmatrix}
            \mathfrak{B}_d^* \mathfrak{P} \mathfrak{B}_d + \mathfrak{F} & \mathfrak{B}_d^* \mathfrak{A}_d^* \mathfrak{P} \mathfrak{B}_d & \cdots &  \mathfrak{B}_d^* {\mathfrak{A}_d^*}^{N-1} \mathfrak{P} \mathfrak{B}_d \\
            \mathfrak{B}_d^* \mathfrak{P} \mathfrak{A}_d \mathfrak{B}_d & \mathfrak{B}_d^* \mathfrak{P} \mathfrak{B}_d + \mathfrak{F} & \cdots & \mathfrak{B}_d^* {\mathfrak{A}_d^*}^{N-2} \mathfrak{P} \mathfrak{B}_d \\
            \vdots & \vdots & \ddots & \vdots \\
            \mathfrak{B}_d^* \mathfrak{P} {\mathfrak{A}_d}^{N-1} \mathfrak{B}_d & \mathfrak{B}_d^* \mathfrak{P} {\mathfrak{A}_d}^{N-2} \mathfrak{B}_d & \cdots & \mathfrak{B}_d^* \mathfrak{P} \mathfrak{B}_d + \mathfrak{F}
        \end{bmatrix} \\
        P = &\begin{bmatrix}
            \mathfrak{B}_d^* \mathfrak{P} {\mathfrak{A}_d} &
            \mathfrak{B}_d^* \mathfrak{P} {\mathfrak{A}_d}^{2}  &
            \hdots &
            \mathfrak{B}_d^* \mathfrak{P} {\mathfrak{A}_d}^{N} 
        \end{bmatrix}^T \\
        T_u (\cdot) = &\begin{bmatrix}
            \langle {\mathfrak{A}_d}^{N} (\cdot), \underline{\phi_u} \rangle
        \end{bmatrix} \\
        S_u = &\begin{bmatrix}
            \langle {\mathfrak{A}_d}^{N-1} \mathfrak{B}_d, \underline{\phi_u} \rangle & 
            \langle {\mathfrak{A}_d}^{N-2} \mathfrak{B}_d, \underline{\phi_u} \rangle &
            \hdots &
            \langle \mathfrak{B}_d, \underline{\phi_u} \rangle
        \end{bmatrix} \\
        U = &\begin{bmatrix}
            u(k+1|k) & u(k+2|k) & \hdots & u(k+N|k)
        \end{bmatrix}^T
    \end{aligned}
\end{equation}