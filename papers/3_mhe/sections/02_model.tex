\section{MODEL REPRESENTATION} \label{sec:2_model}

\subsection{Non-linear System Model with State Delays}

\begin{figure}[!htbp]
    \centering
    \begin{tikzpicture}
        \node (pfr) [cylinder, draw, minimum height=6.5cm, minimum width=1.5cm, shape aspect=1, shape border rotate=180, cylinder uses custom fill, cylinder end fill=green!45, cylinder body fill=green!15] {$\mathcal{A} \rightarrow \mathcal{B}$};
        \node (pfr_inlet) [circle, at={(pfr.west)}, shift={(0.25cm,0)}, fill=blue, draw=blue, inner sep=0pt, minimum size=0.25cm, scale=0.5] {};
        \node (pfr_outlet) [circle, at={(pfr.east)}, shift={(-0.25cm,0)}, fill=blue, draw=blue, inner sep=0pt, minimum size=0.25cm, scale=0.5] {};
        \node (recycle_right) [circle, at={(pfr_outlet.east)}, shift={(1.6cm,0)}, fill=blue, draw=blue, inner sep=0pt, minimum size=0.25cm, scale=0.5] {};
        \node (recycle_left) [circle, at={(pfr_inlet.west)}, shift={(-1.6cm,0)}, fill=blue, draw=blue, inner sep=0pt, minimum size=0.25cm, scale=0.5] {};

        \draw[dotted, thick] ([yshift=0.8cm]pfr_inlet.center) -- node[at end, below, yshift=0.1cm] {$z = 0$} ([yshift=-0.95cm]pfr_inlet.center);
        \draw[dotted, thick] ([yshift=0.8cm]pfr_outlet.center) -- node[at end, below, yshift=0.1cm] {$z = l$} ([yshift=-0.95cm]pfr_outlet.center);

        \node[below of=recycle_left, node distance=2.10cm, anchor=north west, xshift=-0.2cm] {$r_r \, C(l, t_m - \textcolor{red!60!gray}{\boldsymbol{\tilde{t}_r}})$};
        \node[below of=recycle_left, node distance=2.70cm, anchor=north west, xshift=-0.2cm] {$r_r \, T(l, t_m - \textcolor{red!60!gray}{\boldsymbol{\tilde{t}_r}})$};
        \node[above of=pfr_inlet, anchor=south east, node distance=1.30cm, xshift=0.25cm] {$C(0, t_m)$};
        \node[above of=pfr_inlet, anchor=south east, node distance=0.70cm, xshift=0.25cm] {$T(0, t_m)$};
        \node[above of=pfr_outlet, anchor=south west, node distance=1.30cm, xshift=-0.25cm] {$C(l, t_m)$};
        \node[above of=pfr_outlet, anchor=south west, node distance=0.70cm, xshift=-0.25cm] {$T(l, t_m)$};

        \draw [arrow_2] (pfr_outlet) -- node[near end, above] {$\mathbf{y(t_m)} = T(l, t_m)$} ++(4,0);
        \draw [arrow_2] (pfr_inlet) ++(-4,0) coordinate(start) -- node[near start, above, shift={(-0.35cm,0)}] {$C_{Feed}$} (pfr_inlet);
        \draw [arrow_2] (pfr_inlet) ++(-4,0) coordinate(start) -- node[near start, below, shift={(-0.35cm,0)}] {$T_{Feed}$} (pfr_inlet);
        \draw [arrow_2] (recycle_right) -- ++(0,-2.0) -| (recycle_left);

% Heat exchange arrows (from 1.5cm to 1.0cm above reactor)
        \foreach \i in {1,...,17} {
            \draw[->, red!70!black, thick] 
                ($([yshift=1.25cm]pfr_inlet.center) + (0.3333 * \i cm, 0)$) 
                -- 
                ($([yshift=0.75cm]pfr_inlet.center) + (0.3333 * \i cm, 0)$);
        }
        \node[anchor=south, yshift=1.2cm] at (pfr.center) {$\mathbf{u(t_m)} = T_c(t_m)$};
        \node[anchor=south, yshift=1.8cm] at (pfr.center) {Heating/Cooling};

    \end{tikzpicture}
    \caption{Non-isothermal axial dispersion tubular reactor with recycle stream.}
    \label{fig:reactor_scheme}
\end{figure}


A non-isothermal axial dispersion tubular reactor subject to first-order exothermic reaction and partial recycle is considered. The process configuration is shown in Figure~\ref{fig:reactor_scheme}. The governing equations describe the evolution of reactant concentration and temperature profiles along the reactor length, and are obtained by applying standard mass and energy balances over an infinitesimal axial segment \autocite{levenspiel1998chemical}, resulting in a coupled nonlinear convection--diffusion--reaction PDE system given in Equation~(\ref{eq:PDE_1}). Here, $C(z, t_m)$ and $T(z, t_m)$ describe the concentration and temperature profiles along the reactor length $z \in [0, l]$ at time $t_m \in [0, \infty)$, respectively. All variables and parameters are defined in the nomenclature.

\begin{equation} \label{eq:PDE_1}
\begin{cases}
\begin{alignedat}{2}
    \partial_{t_m} C(z, t_m) = &\hspace{1em} D \partial_{zz} C(z, t_m)              &&- v \partial_{z} C(z, t_m) -                              k e^{\dfrac{-E}{R T(z, t_m)}} C(z, t_m)\\[2.0ex]
    \partial_{t_m} T(z, t_m) = &\dfrac{\kappa}{\rho_f c_p} \partial_{zz} T(z, t_m) &&- v \partial_{z} T(z, t_m) - \dfrac{\Delta H}{\rho_f c_p} k e^{\dfrac{-E}{R T(z, t_m)}} C(z, t_m)\\[1.2ex]
    &&& \hspace{3.8em} + \dfrac{4h}{\rho_f c_p d_t} \bigl( T_c(t_m) - T(z, t_m) \bigr) \\
\end{alignedat}
\end{cases}
\end{equation}

The boundary conditions adopt a Danckwerts-type formulation, which has become standard in the modeling of axial dispersion tubular reactors. Belonging to the class of Robin boundaries, Danckwerts conditions preserve generality while remaining physically interpretable in the context of axial dispersion systems \autocite{Danckwerts1953Continuous}.

Except the delay-induced recycle, the present modeling assumptions leading to the governing equations and boundary structure follow \Citeauthor{Khatibi2021Model}'s contribution\autocite{Khatibi2021Model}. In this work, the effect of delay is incorporated into the same class of boundary conditions, giving rise to a recycle-induced delay term at the reactor inlet, as shown in Equation~\eqref{eq:BC_1}.

\begin{equation} \label{eq:BC_1}
\begin{cases}
    \begin{alignedat}{3}
        \partial_{z} C(0, t_m) &=& \dfrac{v}{D} \quad \Bigl[                           &C(0, t_m) - (1-r_r) C_{Feed} & - r_r C(l, t_m-\tilde{t}_r) \Bigr] & \\[1.4ex]
        \partial_{z} T(0, t_m) &=& \hspace{0.3em} \dfrac{\rho_f v c_p}{\kappa} \Bigl[ &T(0, t_m) - (1-r_r) T_{Feed} & - r_r T(l, t_m-\tilde{t}_r) \Bigr] & \\[1.4ex]
        \partial_{z} C(l, t_m) &=& 0 \hspace{0.9cm}                                    &                           &                                  & \\[1.4ex]
        \partial_{z} T(l, t_m) &=& 0 \hspace{0.9cm}                                    &                           &                                  & \\
    \end{alignedat}
\end{cases}
\end{equation}

\subsection{State Delays as Transport PDEs}

As will be discussed in Section~\ref{sec:3_LTI}, the application of infinite-dimensional linear system theory relies on the existence of a strongly continuous semigroup generator acting on a Hilbert space. The delay term in Equation~\eqref{eq:BC_1} introduces a dependence on past values of the system states—i.e., a state delay—which obstructs this formulation and prevents the system from being posed as a standard evolution problem \autocite{Curtain2020Introduction}. To obtain a time-invariant representation suitable for operator-theoretic analysis, the delay terms are replaced by a set of first-order hyperbolic transport PDEs. This approach avoids direct use of delay differential equations by embedding the delay into the state space itself \autocite{Krstic2009Delay}. The resulting formulation follows the one originally developed for an isothermal recycle reactor configuration, and has since been reused in discrete-time controller and observer designs \autocite{moadeli2025optimal,Moadeli2025Model,Moadeli2025Observer}.

To eliminate the explicit delay terms from the reactor boundary conditions, two auxiliary states $C_r(z_r, t_m)$ and $T_r(z_r, t_m)$ are introduced to describe the convective transport through the recycle line. The state delay is thereby reformulated as a transport PDE evolving over the pseudo-spatial domain $z_r \in [0, l_r]$, governed by:

\begin{equation} \label{eq:transport_PDE}
    \begin{cases}
        \partial_{t_m} C_r(z_r, t_m) = - v_r \, \partial_{z_r} C_r(z_r, t_m) \\
        \partial_{t_m} T_r(z_r, t_m) = - v_r \, \partial_{z_r} T_r(z_r, t_m)
    \end{cases}
\end{equation}

with boundary conditions:

\begin{equation} \label{eq:transport_BC}
    \begin{cases}
        C_r(l_r, t_m) = C(l, t_m) \\
        T_r(l_r, t_m) = T(l, t_m)
    \end{cases}
\end{equation}

Evaluating the transport states at the inlet, the delayed boundary terms in Equation~\eqref{eq:BC_1} are equivalently expressed as:
\begin{equation} \label{eq:delay_identity}
    \begin{cases}
        C_r(0, t_m) = C_r(l_r, t_m - \tilde{t}_r) = C(l, t_m - \tilde{t}_r) \\
        T_r(0, t_m) = T_r(l_r, t_m - \tilde{t}_r) = T(l, t_m - \tilde{t}_r)
    \end{cases}
\end{equation}

The substitution of Equations~\eqref{eq:transport_PDE}-\eqref{eq:delay_identity} into the original system in Equations~\eqref{eq:PDE_1} and~\eqref{eq:BC_1} yields an equivalent formulation composed of four coupled nonlinear PDEs. This time-invariant representation, summarized in Equations~\eqref{eq:PDE_2} and~\eqref{eq:BC_2}, is a necessary step toward enabling the application of infinite-dimensional system theory via late-lumping, where the system must admit a well-posed Cauchy problem governed by a strongly continuous semigroup on a Hilbert space.


\begin{equation} \label{eq:PDE_2}
\begin{cases}
\begin{alignedat}{2}
    \partial_{t_m} C(z, t_m) = &\hspace{1.5em} D \partial_{zz} C(z, t_m) 
    &&- v \partial_z C(z, t_m) 
    - k e^{\dfrac{-E}{R T(z, t_m)}} C(z, t_m) \\[2.0ex]

    \partial_{t_m} T(z, t_m) = &\hspace{1em}\dfrac{\kappa}{\rho_f c_p} \partial_{zz} T(z, t_m) 
    &&- v \partial_z T(z, t_m)
    - \dfrac{\Delta H}{\rho_f c_p} k e^{\dfrac{-E}{R T(z, t_m)}} C(z, t_m) \\[1.2ex]
    &&& \hspace{4.1em} + \dfrac{4h}{\rho_f c_p d_t} \bigl( T_c(t_m) - T(z, t_m) \bigr) \\[2.0ex]

    \partial_{t_m} C_r(z_r, t_m) = &- v_r \partial_{z_r} C_r(z_r, t_m) \\[2.0ex]

    \partial_{t_m} T_r(z_r, t_m) = &- v_r \partial_{z_r} T_r(z_r, t_m)
\end{alignedat}
\end{cases}
\end{equation}

\begin{equation} \label{eq:BC_2}
\begin{cases}
\begin{alignedat}{2}
    \partial_z C(0, t_m) &= \dfrac{v}{D} \left[ C(0, t_m) - (1 - r_r) C_\text{Feed} - r_r C_r(0, t_m) \right], \quad &\partial_z C(l, t_m) &= 0 \\[1.5ex]

    \partial_z T(0, t_m) &= \dfrac{\rho_f v c_p}{\kappa} \left[ T(0, t_m) - (1 - r_r) T_\text{Feed} - r_r T_r(0, t_m) \right], \quad &\partial_z T(l, t_m) &= 0 \\[1.5ex]

    C_r(l_r, t_m) &= C(l, t_m) \\[1.5ex]
    T_r(l_r, t_m) &= T(l, t_m)
\end{alignedat}
\end{cases}
\end{equation}

For notational simplicity and to reveal the structure of the system more clearly, a dimensionless formulation is adopted using reference inlet values and characteristic length and time scales. The transformation is defined in Equation~\eqref{eq:dimless_transformation}, leading to the introduction of normalized spatial and temporal coordinates, as well as dimensionless state variables. 

\begin{equation} \label{eq:dimless_transformation}
\begin{cases}
    \begin{alignedat}{2}
        &\zeta = \dfrac{z}{l} = \dfrac{z_r}{l_r}, \qquad 
        t = \dfrac{t_m}{\tilde{t}}, \qquad
        &&\tau = \dfrac{\tilde{t}_r}{\tilde{t}}, \\[2.0ex]
        &m_1(\zeta, t) = \dfrac{C_{\text{Feed}} - C(\zeta, t)}{C_{\text{Feed}}}, \quad 
        &&m_2(\zeta, t) = \dfrac{T(\zeta, t) - T_{\text{Feed}}}{T_{\text{Feed}}}, \\[1.5ex]
        &m_3(\zeta, t) = \dfrac{C_{\text{Feed}} - C_r(\zeta, t)}{C_{\text{Feed}}}, \quad 
        &&m_4(\zeta, t) = \dfrac{T_r(\zeta, t) - T_{\text{Feed}}}{T_{\text{Feed}}}, \\[1.5ex]
        &T_w(t) = \dfrac{T_c(t) - T_{\text{Feed}}}{T_{\text{Feed}}}
    \end{alignedat}
\end{cases}
\end{equation}

Substituting these variables into the nonlinear PDE system in Equations~\eqref{eq:PDE_2}-\eqref{eq:BC_2} yields the dimensionless representation of the system given in Equations~\eqref{eq:PDE_3} and~\eqref{eq:BC_3}. 


\begin{equation} \label{eq:PDE_3}
\begin{cases}
\begin{alignedat}{2}
    \partial_{t} m_1(\zeta, t) = &\dfrac{1}{Pe_m} \partial_{\zeta\zeta} m_1(\zeta, t) 
    -\partial_{\zeta} m_1(\zeta, t) 
    &&+ k_a \Bigl( 1 - m_1(\zeta, t) \Bigr) e^{\frac{\eta m_2(\zeta, t)}{1 + m_2(\zeta, t)}} \\[2.0ex]

    \partial_{t} m_2(\zeta, t) = &\dfrac{1}{Pe_T} \partial_{\zeta\zeta} m_2(\zeta, t) 
    -\partial_{\zeta} m_2(\zeta, t)
    &&+ \alpha k_a \Bigl( 1 - m_1(\zeta, t) \Bigr) e^{\frac{\eta m_2(\zeta, t)}{1 + m_2(\zeta, t)}} \\[1.2ex]
    &&&+ \sigma \Bigl( T_w(t) - m_2(\zeta, t)\Bigr) \\[2.0ex]

    \partial_{t} m_3(\zeta, t) = &\frac{1}{\tau} \partial_{\zeta} m_3(\zeta, t) \\[2.0ex]

    \partial_{t} m_4(\zeta, t) = &\frac{1}{\tau} \partial_{\zeta} m_4(\zeta, t)
\end{alignedat}
\end{cases}
\end{equation}

\begin{equation} \label{eq:BC_3}
\begin{cases}
\begin{alignedat}{2}
    \partial_\zeta m_1(0, t) &= Pe_m \left[ m_1(0, t) - r_r m_3(0, t) \right], \quad &\partial_{\zeta} m_1(l, t) &= 0 \\[1.5ex]

    \partial_\zeta m_2(0, t) &= Pe_T \left[ m_2(0, t) - r_r m_4(0, t) \right], \quad &\partial_{\zeta} m_2(l, t) &= 0 \\[1.5ex]

    m_1(1, t) &= m_3(1, t) \\[1.5ex]

    m_2(1, t) &= m_4(1, t)
\end{alignedat}
\end{cases}
\end{equation}


\subsection{Steady-State Analysis}

The steady-state configuration of the system is obtained by setting all time derivatives in Equation~\eqref{eq:PDE_2} to zero, yielding a system of four coupled nonlinear ODEs. The resulting boundary value problem is solved numerically using a collocation method with adaptive mesh refinement, as implemented in standard boundary value solvers. The solution process involves discretizing the spatial domain, minimizing the residual of the governing equations subject to nonlinear boundary conditions, and iteratively converging to a consistent steady-state profile.

Due to the nonlinear coupling in the reaction and energy terms, the system may exhibit multiple equilibrium profiles. This is well-established in the context of exothermic tubular reactors, where the interdependence between the concentration and temperature fields can generate both stable and unstable steady states \autocite{Heinemann1982effect, Hastir2020Analysis}.

Similar to the results obtained in \autocite{Khatibi2021Model}, one parameter configuration gives rise to multiple steady-state profiles, while a second set leads to a unique stable solution. These are shown in Figures~\ref{fig:ss_profiles}~and~\ref{fig:ss_profiles_stable}, respectively. The two parameter sets are listed side-by-side in Table~\ref{tab:pars}, and differ only in the temperature dependence of the reaction kinetics. The unstable equilibrium from the first case and the stable equilibrium from the second will serve as linearization points in the control and estimation developments that follow.

\begin{figure}[!htbp]
    \centering
    \includesvg[inkscapelatex=false, width=1.0\textwidth, keepaspectratio]{papers/3_mhe/figures/steady_state_I.svg}
    \caption{Steady-state solutions for Case I. Solid and dashed lines represent reactor and recycle stream profiles, respectively.}
    \label{fig:ss_profiles}
\end{figure}

\begin{figure}[!htbp]
    \centering
    \includesvg[inkscapelatex=false, width=1.0\textwidth, keepaspectratio]{papers/3_mhe/figures/steady_state_II.svg}
    \caption{Steady-state solution for Case II. Solid and dashed lines represent reactor and recycle stream profiles, respectively.}
    \label{fig:ss_profiles_stable}
\end{figure}

\begin{table}[htbp]
\centering
\renewcommand{\arraystretch}{1.3}
% \setlength{\tabcolsep}{10pt}
\caption{Parameters used in the steady-state analysis for Case I (Unstable) and Case II (Stable)} \label{tab:pars}
\begin{tabular}{|c|c|c|}
\hline
\textbf{Parameter} & \makecell{\textbf{Case I} \\ \textbf{(Unstable)}} & \makecell{\textbf{Case II} \\ \textbf{(Stable)}} \\
\hline
$\mathrm{Pe}_m$  & 4        & 4      \\
$\mathrm{Pe}_T$  & 6        & 6      \\
$T_{\mathrm{w}}^{\mathrm{ss}}$ & -0.37 & -0.37  \\
$T_{\mathrm{feed}}$ & 600 K & 600 K  \\
$C_{\mathrm{feed}}$ & 1.0 M & 1.0 M  \\
$k_a$            & 0.6      & 0.6    \\
$r_r$            & 0.3      & 0.3    \\
$\alpha$         & 0.8      & 0.8    \\
$\eta$           & 14.0     & 6.0    \\
$\sigma$         & 0.9      & 0.9    \\
$\tau$           & 0.5      & 0.5    \\
$R_1$            & -1.38    & -0.45  \\
$R_2$            & 6.48     & 1.95   \\
\hline
\end{tabular}
\end{table}


\subsection{Linearized Model}

To facilitate controller and observer design, the dimensionless nonlinear system in Equations~\eqref{eq:PDE_3}-\eqref{eq:BC_3} is linearized around a chosen steady-state profile. Deviation variables are introduced as $ x_i(\zeta, t) = m_i(\zeta, t) - m_i^{ss}(\zeta) $, where $ m_i^{ss} $ denotes the corresponding steady-state solution. These deviation variables form the state vector of the linearized system. 

The jacket inlet temperature $ T_w(t) $ is treated as the manipulated input and is similarly expressed in deviation form as $ u(t) = T_w(t) - T_w^{ss} $. The measured output is taken to be the reactor outlet temperature deviation, defined as $ y(t) = m_2(1, t) - m_2^{ss}(1) $.

The nonlinear reaction source terms, which depend on both concentration and temperature, are linearized with respect to the deviation variables:

\begin{equation}
\begin{cases}
    f_{\text{nl}}(m_1, m_2) = k_a (1 - m_1) e^{\frac{\eta m_2}{1 + m_2}} \approx f_{\text{nl}}(m_{1}^{ss}, m_{2}^{ss}) + \tilde{R}_1 (m_1 - m_{1}^{ss}) + \tilde{R}_2 (m_2 - m_{2}^{ss}), \\[1.5ex]
    g_{\text{nl}}(m_1, m_2) = \alpha k_a (1 - m_1) e^{\frac{\eta m_2}{1 + m_2}} \approx g_{\text{nl}}(m_{1}^{ss}, m_{2}^{ss}) + \alpha \tilde{R}_1 (m_1 - m_{1}^{ss}) + \alpha \tilde{R}_2 (m_2 - m_{2}^{ss})
\end{cases}
\end{equation}


where the local Jacobian terms are given by:

\begin{equation}
    \tilde{R}_1(\zeta) = -k_a e^{\frac{\eta m_{2}^{ss}(\zeta)}{1 + m_{2}^{ss}(\zeta)}}, \quad
    \tilde{R}_2(\zeta) = \frac{\eta k_a (1 - m_{1}^{ss}(\zeta)) e^{\frac{\eta m_{2}^{ss}(\zeta)}{1 + m_{2}^{ss}(\zeta)}}}{(1 + m_{2}^{ss}(\zeta))^2}.
\end{equation}

To simplify the linearized system, spatially averaged coefficients are introduced:

\begin{equation}
    R_1 = \int_0^1 \tilde{R}_1(\zeta) \, d\zeta, \quad
    R_2 = \int_0^1 \tilde{R}_2(\zeta) \, d\zeta.
\end{equation}

The spatially averaged parameters $R_1$ and $R_2$ are calculated based on the steady-state profiles obtained in the previous section, with their values listed in Table~\ref{tab:pars} for each parameter configuration. These linearized reaction terms will be incorporated into the governing equations to obtain a spatially-invariant linear PDE model given in Equation~\eqref{eq:PDE_4}. This approximation enables a more tractable representation while preserving the system's infinite-dimensional character for operator-theoretic formulation.

\begin{equation} \label{eq:PDE_4}
\begin{cases}
\begin{alignedat}{3}
    \partial_{t} x_1(\zeta, t) &= \dfrac{1}{Pe_m} \partial_{\zeta\zeta} x_1(\zeta, t) 
    - \partial_{\zeta} x_1(\zeta, t) 
    &+ R_1 x_1(\zeta, t) &+ R_2 x_2(\zeta, t) \\[2.0ex]

    \partial_{t} x_2(\zeta, t) &= \dfrac{1}{Pe_T} \partial_{\zeta\zeta} x_2(\zeta, t) 
    - \partial_{\zeta} x_2(\zeta, t)
    &+ \alpha R_1 x_1(\zeta, t)  &+ \alpha R_2 x_2(\zeta, t) + \sigma \left[ u(t) - x_2(\zeta, t) \right] \\[2.0ex]

    \partial_{t} x_3(\zeta, t) &= \dfrac{1}{\tau} \partial_{\zeta} x_3(\zeta, t) 
    &\quad & \\[2.0ex]

    \partial_{t} x_4(\zeta, t) &= \dfrac{1}{\tau} \partial_{\zeta} x_4(\zeta, t) 
    &\quad & \\[3.0ex]

    \partial_\zeta x_1(0, t) &= Pe_m \left[ x_1(0, t) - r_r x_3(0, t) \right], 
    &\quad \partial_{\zeta} x_1(1, t) &= 0 \\[1.5ex]

    \partial_\zeta x_2(0, t) &= Pe_T \left[ x_2(0, t) - r_r x_4(0, t) \right], 
    &\quad \partial_{\zeta} x_2(1, t) &= 0 \\[1.5ex]

    x_1(1, t) &= x_3(1, t) \\[1.5ex]
    x_2(1, t) &= x_4(1, t) \\[1.5ex]
    y(t) &= x_2(1, t)
\end{alignedat}
\end{cases}
\end{equation}

This linearized representation forms the foundation for the infinite-dimensional state-space modeling, estimation, and control design discussed in the next section.