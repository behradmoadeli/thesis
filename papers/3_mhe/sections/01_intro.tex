\section{INTRODUCTION}

Distributed parameter systems (DPS), typically modeled by partial differential equations (PDEs), arise naturally in chemical engineering applications such as catalytic reactors, heat exchangers, and fluidized beds, where mass and energy transport processes are distributed over space and time \autocite{ray1981advanced, davison1976robust,Curtain2020Introduction}. Among these, axial dispersion tubular reactors with recycle represent an important class of industrial systems, characterized by strong coupling between reaction, convection, diffusion, and recirculation dynamics \autocite{jensen1982bifurcation, Ali2015Review, Hlavacˇek1970Modeling,Hlavacˇek1970Modelinga,Cohen1974Tubular,Georgakis1977Studies}. These reactors are often modeled using second-order parabolic PDEs under Danckwerts boundary conditions to reflect realistic inlet and outlet transport assumptions \autocite{Danckwerts1953Continuous}. The inclusion of a recycle stream—commonly used to enhance conversion or efficiency—adds complexity, including dynamic instability and multiple steady states, under certain operating regimes \autocite{Luss1967Stability,Bildea2004Design}. 

While numerous studies have investigated control of non-isothermal tubular reactors, most have either neglected the effect of recycle or, as in the work of \Citeauthor{Khatibi2021Model}, assumed it to be instantaneous—effectively disregarding the finite residence time of the returning stream \autocite{Khatibi2021Model}. This assumption, although mathematically convenient, limits the realism and applicability of the resulting control strategies. In contrast, recent advances have explored modeling recycle delay as a transport phenomenon. \Citeauthor{moadeli2025optimal} introduced a delay-aware framework for tubular reactors by modeling the recycle loop as a first-order hyperbolic PDE \autocite{moadeli2025optimal}, a formulation that embeds the delay directly into the infinite-dimensional system. This approach is consistent with a well-established tradition in PDE control \autocite{Christofides1997Finite,Armaou2002Dynamic, aksikas2017optimal,balas1979feedback,Curtain1982Finite}, commonly referred to as the late-lumping paradigm, where spatial dynamics are preserved throughout control synthesis to avoid distortions from premature discretization.

Modeling recycle-induced delay as a transport PDE, rather than as a delay differential equation (DDE), unifies and preserves the spatial-temporal structure of the system and enables analysis within an infinite-dimensional control framework \autocite{Krstic2009Delay}. Although rare in chemical engineering applications, such representations have been explored in general DPS literature as alternatives to lumped-delay models, particularly in systems with internal transport structures, giving rise to state delays \autocite{Qi2021Output,Cassol2019Heat}. In \Citeauthor{moadeli2025optimal}'s work, the recycle-induced state delay was treated as a convective domain coupled to a parabolic PDE, leading to a Riesz-spectral generator and allowing for optimal control synthesis using operator Riccati equations \autocite{moadeli2025optimal}. However, that work assumed an isothermal reactor, relied on non-optimal state reconstruction via a simple Luenberger observer, and did not address energy balances, process/measurement noises, or input/state constraints.

The present study unifies two previously disjoint modeling approaches. \Citeauthor{moadeli2025optimal} \autocite{moadeli2025optimal} introduced a delay-aware framework for tubular reactors by modeling the recycle stream as a transport PDE, capturing recycle-induced state delay within an infinite-dimensional formulation, but limited the model to isothermal dynamics. In contrast, \Citeauthor{Khatibi2021Model} \autocite{Khatibi2021Model} addressed non-isothermal reactor behavior but assumed an instantaneous recycle stream, neglecting the delay mechanism altogether. The current study integrates these two formulations by combining the transport-PDE representation of recycle delay with the full mass-energy dynamics of a non-isothermal reactor, resulting in a coupled four-equation PDE model. To prepare this infinite-dimensional system for implementation of digital estimation and control schemes, time discretization is performed using the Cayley-Tustin transformation; i.e. a structure-preserving midpoint integration technique that maintains Hamiltonian structure and improves robustness against sampling distortion\autocite{havu2007cayley, xu2017linear}.

From a control and estimation perspective, each prior effort addressed only part of the broader challenge. \Citeauthor{moadeli2025optimal} formulated both full-state and output-based regulators using infinite-dimensional LQR theory in continuous time \autocite{moadeli2025optimal}, but did not consider constraints or digital implementation. \Citeauthor{Khatibi2021Model}, by contrast, developed a constrained model predictive controller (MPC) in discrete time for a non-isothermal reactor, but employed a simple Luenberger observer for state estimation and neglected the recycle delay \autocite{Khatibi2021Model}. Recent discrete-time implementations based on the delay-aware model \autocite{Moadeli2025Model,Moadeli2025Observer} extended the formulation to output-feedback MPC under constraints, yet continued to rely on Luenberger observers and did not incorporate temperature dynamics. In the current study, the observer is replaced with a moving horizon estimator (MHE), enabling output-based optimal state reconstruction under constraints and integrating it with constrained MPC within the same infinite-dimensional framework, while taking measurement noises as well as plant-model mismatch into account. The MHE formulation is inspired by recent developments in infinite-dimensional estimation for PDE systems \autocite{xie2022constrained, zhang2023tracking}, and is adapted here to address the dynamics of a non-isothermal reactor with recycle-induced state delay.

This work represents the first MHE-MPC architecture to jointly incorporate non-isothermal reaction-diffusion-convection dynamics, recycle-induced state-delays, constrained MPC, and output-based estimation via MHE, all within a late-lumped infinite-dimensional setting. The rest of the paper is organized as follows: Section 2 presents the modeling and system formulation. Section 3 outlines the operator-theoretic representation. Section 4 details Cayley-Tustin discretization. Sections 5 and 6 present the MPC and MHE formulations, respectively. Section 7 illustrates closed-loop performance under realistic constraints and measurement noise.
