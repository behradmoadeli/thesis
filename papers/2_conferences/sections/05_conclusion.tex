\section{Conclusion} \label{sec:conclusion}

In this work, model predictive control of an axial dispersion tubular reactor equipped with recycle is addressed, while considering the delay imposed by the recycle stream. This setup is common in industry but has received limited attention in the chemical engineering distributed parameter systems literature. The diffusion-convection-reaction dynamics of the reactor is modeled by a second-order parabolic PDE, while a notion of state delay is introduced to account for the delay imposed by the recycle stream. The state delay is addressed as a separate transport PDE, resulting in a boundary-controlled system governed by a coupled set of parabolic and hyperbolic PDEs under Danckwerts boundary conditions. Utilizing a late-lumping approach, the resolvent operator is obtained in a closed form in order to preserve the infinite-dimensional nature of the system without requiring spatial discretization. To implement MPC as a digital controller, the Cayley–Tustin transformation is used. This Crank–Nicolson type of discretization is chosen as it maintains important properties of the system such as stability and controllability when mapping the continuous-time system to a discrete-time one. Numerical simulations demonstrate the effectiveness of the proposed controller in stabilizing an unstable system while satisfying input constraints under full-state feedback. Recognizing, however, that full-state information is often unavailable in practical implementations of distributed parameter systems, this work is further extended by designing and integrating a discrete-time Luenberger observer to reconstruct the state from output measurements alone, without any spatial approximation. A family of observer gains is examined, and spectral analysis is performed to select gains that ensure the state reconstruction error converges faster than the closed-loop system dynamics. This guarantees accurate state estimates during transients and prevents performance degradation due to estimation delay. The proposed approach can be further extended to incorporate the effects of temperature as well as disturbance rejection or set-point tracking in future work.