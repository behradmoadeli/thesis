\section{Introduction}

Many chemical and petrochemical processes, such as reactions in tubular reactors, heat transfer in exchangers, and separations in columns, involve states distributed in space and time. These systems, known as distributed parameter systems (DPS), are often modeled using partial differential equations (PDEs) to describe distributed state dynamics. Due to their infinite-dimensional nature, the control and estimation of DPSs are inherently more challenging compared to the well-established control theories for finite-dimensional systems~\cite{ray1981advanced}, making this field an active area of research. Two primary methods, ``Early Lumping'' and ``Late Lumping,'' have been proposed to address DPS control in the literature. The first, ``Early Lumping,'' reduces the infinite-dimensional system to a finite-dimensional one through spatial discretization during the modeling phase~\cite{davison1976robust}. While this enables standard control strategies, it often compromises model accuracy due to mismatches between the original and reduced-order systems~\cite{moghadam2012infinite}. In contrast, ``Late Lumping'' preserves the infinite-dimensional system until the final numerical implementation stage, resulting in more accurate but computationally complex control strategies~\cite{ray1981advanced}.

State reconstruction for DPSs has also been addressed using discrete-time Luenberger observers without spatial discretization, a key feature consistent with the late lumping paradigm~\cite{dochain2000state,dochain2001state,alonso2004optimal,Ali2015Review}.
Numerous studies have employed Late Lumping approaches to control infinite-dimensional systems in the field of chemical engineering. These efforts primarily focus on convection-reaction systems governed by first-order hyperbolic PDEs and diffusion-convection-reaction systems governed by second-order parabolic PDEs. For example, robust control of first-order hyperbolic PDEs was explored in~\cite{christofides1996feedback}, where a plug flow reactor system was stabilized under distributed input. Similarly, boundary feedback stabilization using the backstepping method was proposed in~\cite{krstic2008backstepping} for such systems. State feedback regulator design for a countercurrent heat exchanger, another example of a chemical engineering DPS, was addressed in~\cite{xu2016state}. Introducing the effects of dispersion in tubular reactors, robust control of second-order parabolic PDEs was studied in~\cite{christofides1998robust}. Modal decomposition methods for designing low-dimensional predictive controllers for diffusion-convection-reaction systems have also been applied in~\cite{dubljevic2006predictive}, while observer-based model predictive control (MPC) was developed in~\cite{Khatibi2021Model} for axial dispersion tubular reactors, considering recycle stream effects.

Delay systems represent another class of infinite-dimensional systems studied extensively~\cite{Curtain2020Introduction}. Commonly modeled using delay differential equations (DDEs), delays can alternatively be described using transport PDEs, offering advantages in complex scenarios~\cite{Krstic2009Delay}. In chemical engineering DPS control, input/output delays have been widely addressed, as industrial processes often feature both measurement and actuation delays. Such delays are typically handled by modeling them as transportation lag blocks, resulting in cascade PDE systems~\cite{hiratsuka1969optimal,Mohammadi2012LQ,Cassol2019Discrete}. State delays, though less common, have been investigated in specific applications, such as heat exchangers with stream delays between passes~\cite{Cassol2019Heat}, and plug flow tubular reactors with recycle delays~\cite{Qi2021Output}; with the effect of dispersion not being addressed in any of these works. Even in~\cite{Khatibi2021Model}, where the effect of recycle is studied for an axial dispersion tubular reactor, the recycle is assumed to be instantaneous, leaving a gap in the literature regarding state delays in diffusion-convection-reaction systems with recycle streams.

In this work, an axial dispersion reactor with recycle is modeled as a diffusion-convection-reaction DPS. The reactor dynamics are described by a second-order parabolic PDE, coupled with a first-order hyperbolic transport PDE to account for the recycle stream’s state delay. A Late Lumping approach is employed, obtaining the system’s resolvent in a closed operator form without spatial discretization. To implement MPC as a digital controller, the system is discretized using the Cayley-Tustin method, a Crank-Nicolson-type discretization that conserves the continuous system’s characteristics, avoiding the need for model reduction. Numerical simulations demonstrate that the proposed controller stabilizes an unstable system optimally under input constraints. A discrete-time infinite-dimensional Luenberger observer is designed to reconstruct unmeasured states, enabling output feedback MPC. Simulations show that the proposed controller successfully stabilizes the otherwise unstable system under input constraints.
