\abstracttext{%
	Many chemical processes identified as distributed parameter systems (DPSs) are described by partial differential equations (PDEs), which pose significant challenges for control design due to their infinite-dimensional nature. One prominent example is an axial dispersion tubular chemical reactor with an internal recycle loop—a system that, as shown in this thesis, exhibits an intrinsic state delay not previously recognized in the chemical engineering DPS literature. This doctoral research develops a comprehensive control and estimation framework that explicitly addresses this delay by reformulating the reactor’s dynamics into a coupled PDE model, capturing the recycle-induced delay as a transport PDE alongside the reactor’s convection–diffusion–reaction PDE to preserve the system’s infinite-dimensional structure. Using this first-principles model as a foundation, advanced control and estimation techniques are designed in a late-lumped manner—meaning the distributed model form is retained until the final numerical implementation. In particular, an infinite-dimensional linear quadratic regulator (LQR) and a model predictive control (MPC) scheme are developed to stabilize the inherently unstable reactor. To enable output-feedback control with limited sensing, a Luenberger observer and a moving horizon estimator (MHE) are integrated, allowing state reconstruction and effective operation in the presence of measurement noise and input constraints. Simulation studies demonstrate that the proposed delay-aware framework successfully stabilizes the inherently unstable reactor and meets key control objectives even in a challenging non-isothermal (exothermic) case. Overall, this thesis establishes a novel methodology that bridges theoretical optimal control of PDEs with practical chemical reactor systems, providing a generalizable foundation for controlling other processes with state delays in the field of Chemical Engineering distributed parameter systems.
}