\chapter{Concluding Remarks}\label{ch:4}
	\section{Conclusion}
		This thesis has presented a unified modeling and control framework for distributed parameter systems with internal state delays, exemplified by an axial dispersion tubular reactor with a recycle loop. All the research objectives outlined in \Cref{ch:0} were successfully achieved. The existence of a recycle-induced state delay was formally identified and incorporated into the reactor model by introducing an additional transport PDE, thereby converting the system into a coupled parabolic–hyperbolic PDE model without any ad-hoc simplifications. The reactor model—chosen not as a special case but as a physically grounded and broadly representative DPS—captures key features common to many industrial processes and provides a realistic yet general testbed. This inclusive modeling approach preserved the infinite-dimensional structure of the problem and laid the groundwork for advanced, late-lumped control design.

		On this foundation, a series of estimation and control strategies were developed and validated through simulation studies. First, in the isothermal reactor scenario, an infinite-dimensional linear quadratic regulator (LQR) was designed to stabilize the system, and a state observer was implemented to enable output feedback using only boundary measurements. Next, the framework was translated to the digital domain: using a structure-preserving Cayley–Tustin time discretization, a model predictive control (MPC) scheme was formulated to maintain stability while explicitly handling input constraints. A discrete-time Luenberger observer was integrated with the MPC to reconstruct the system state from limited outputs, demonstrating effective output-feedback control in the isothermal case. Finally, the methodology was extended to a more complex non-isothermal (exothermic) reactor. There, a moving horizon estimator (MHE) was paired with MPC to achieve constrained control under realistic conditions of process and measurement noise. Across these studies, the proposed controllers successfully stabilized otherwise unstable reactor conditions and met key performance criteria, confirming the viability of the late-lumping approach for delay-affected DPSs.

		In summary, this thesis revealed that a common chemical engineering system—the tubular reactor with recycle—exhibits a state delay that had not been formally addressed in previous DPS literature. A structure-preserving model was developed, and advanced estimation and control strategies were designed and implemented using a late-lumped approach. The results demonstrated that delay-aware controllers can stabilize an inherently unstable reactor while retaining physical fidelity, even under non-isothermal conditions. These contributions provide a concrete step toward bringing modern control methods into practical use for distributed systems with internal delays.

	\section{Future Scope}
		While this thesis focused on core theoretical development under nominal conditions, it opens several avenues for future research and application:
		\begin{itemize}
			\item \textbf{Robustness and Uncertainty:} Future studies should incorporate model uncertainties, unmodeled dynamics, and time-varying parameters into the framework. Developing robust or adaptive controllers (and observers) will ensure that the delay-aware strategy remains effective under parameter drift and disturbances, extending its reliability in industrial scenarios.
			\item \textbf{Enhanced Control Objectives:} Building on the stabilization achieved here, the control scheme can be extended to address objectives like set-point tracking and disturbance rejection. This would involve augmenting the current regulators or MPC designs to handle changing reference signals and to actively compensate for external perturbations, thereby broadening the framework’s applicability in plant operations.
			\item \textbf{Input/Output Delay Integration:} The present work dealt with an intrinsic state delay, but many processes also experience input delays (in actuators) and output delays (in sensors). Incorporating such delays into the model and control design – alongside the state delay – is a natural extension, as the same transport PDE framework can accommodate these additional dynamics. This would yield a more comprehensive delay-aware control strategy that reflects real-world conditions more accurately.
			\item \textbf{Application to Other Systems:} The delay-aware, PDE-based approach can be generalized to other chemical engineering systems or process networks featuring recirculation or transport delays (e.g. recycle streams in multi-unit processes, pipeline networks, or large-scale reactors). Applying the framework to these systems will test its generality and could yield tailored strategies for various industries.
			\item \textbf{Experimental Validation:} Finally, implementing the proposed estimation and control schemes in a laboratory or pilot-scale reactor setup is an important step toward practical adoption. In addition to revealing implementation challenges, such an experimental program would enable meaningful benchmarking against heuristic and early-lumped controllers, where the plant behavior is no longer constrained by a specific finite-dimensional approximation. This would provide clearer insight into the real-world efficacy of the late-lumped, state-delay-inclusive control approach.
		\end{itemize}
