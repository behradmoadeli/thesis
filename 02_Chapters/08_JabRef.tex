\chapter{JabRef: Managing Bibliographies Efficiently}\label{ch:JabRef}
	\section{Introduction}\label{sec:JRIntroduction}
		JabRef stands as a powerful tool for researchers and academics engaged in scholarly writing.
		JabRef offers a robust solution for bibliography management, including a number of features to ensure that you are not only able to organize your references but keep track of progress, and notes on each reference.
		This chapter aims to provide an review of JabRef, including its myriad features that I find particularly useful, and to guide you through its implementation in writing your thesis.

	\section{Key Features of JabRef}
		JabRef, with its versatile features, emerges as an indispensable tool for bibliography/reference management. 
		Delving deeper into its functionalities reveals a wealth of tools designed to streamline the often cumbersome process of handling references. 

		\subsection{BibTeX Compatibility}
			JabRef's commitment to the BibTeX format is a testament to its roots in the \LaTeX\ ecosystem. 
			This compatibility ensures a seamless integration between the reference management process and the \LaTeX\ document preparation workflow. 
			Users can easily export and import BibTeX files, facilitating collaboration and compatibility across various platforms.

		\subsection{Reference Import}
			The capability to import references directly from online databases and journal websites significantly accelerates the reference collection process. 
			JabRef supports various import formats, allowing users to effortlessly populate their databases with accurate and structured reference information. 
			This feature is particularly valuable for researchers dealing with large/extensive bibliographies.

		\subsection{Customizable Entry Types}
			The flexibility offered by customizable entry types allows users to categorize references based on the nature of the source. 
			Whether it's a book, article, conference proceeding, or any other reference type, JabRef accommodates diverse sources, ensuring a well-organized and easily navigable bibliography.

		\subsection{Search and Filter}
			The ability to efficiently search and filter references is a hallmark of JabRef's usability. 
			Researchers dealing with extensive databases will appreciate the quick and precise retrieval of references based on author names, titles, keywords, or any other criteria. 
			This feature is crucial for maintaining order in a rapidly growing bibliography.

		\subsection{Grouping}
			JabRef's grouping functionality provides a systematic approach to organizing references. 
			Users can create custom groups to categorize references based on themes, projects, or any other criteria. 
			This feature is especially useful for large research projects where a systematic organization of references is essential for maintaining clarity and coherence.

		\subsection{Integration with \LaTeX}
			The seamless integration of JabRef with \LaTeX\ editors fortifies the synergy between bibliography management and document preparation. 
			This integration minimizes the manual effort required for citation insertion and ensures consistency between the bibliography and the in-text citations. 
			Users can easily copy citation keys from JabRef and paste them directly into their \LaTeX\ documents.

	\section{Getting Started with JabRef}
		Now that we've outlined the key features of JabRef, let's embark on a comprehensive guide on how to get started with JabRef. 
		This step-by-step walkthrough will cover everything from installation to creating a new bibliography and populating it with references.

		\subsection{Installation}
			The initial step in utilizing JabRef is to install the software on your system. 
			For all users, regardless of OS, the easiest way to download JabRef is to visit their website: \url{https://www.jabref.org/}.
			Once there select `Download' from the navigation bar, and press the ``Download JabRef'' button. 
			This will take you to the FossHub page where you can select the appropriate version for your OS and download and install it.

		\subsection{Creating a New Bibliography}
			Once JabRef is successfully installed, launch the application. 
			When the program loads you will be faced with a window that looks like the one shown in \Cref{fig:JabRefProgram}.
			\begin{figure}[htbp]
				\centering
				\includegraphics[width=0.7\linewidth]{\insertimage{JabRef_Main_Window.png}}
				\caption{JabRef Main Window.}
				\label{fig:JabRefProgram}
			\end{figure}
			Now that the program is open, to create a new bibliography:
			\begin{enumerate}
				\item Click on `File $\rightarrow$ New Library'.
			\end{enumerate}
			To save the database:
			\begin{enumerate}
				\item Click on `File $\rightarrow$ Save Library'.
				\item Choose an appropriate and location.
				\item Click `Save'.
			\end{enumerate}
			Congratulations! You've initiated your bibliography using JabRef. 
			Now that we have this created, the next step is to add references to the database.

		\subsection{Adding References}
			JabRef offers multiple avenues for adding references to your database.
			Some of the methods are generally more useful than others but we will go over a few that you are likely to use:

			\subsubsection{Web Search}
				JabRef's integrated web search (see \Cref{fig:JabRefWebSearch}) feature simplifies the process of importing references from online sources.
				This is by-far the easiest way to enter a reference.
				
				\begin{figure}[htbp]
					\centering
					\includegraphics[width=0.35\linewidth]{\insertimage{JabRef_Web_Search.png}}
					\caption{JabRef Web Search Tool.}
					\label{fig:JabRefWebSearch}
				\end{figure}
				
				\begin{enumerate}
					\item Click on `Web Search'.
					\item Search for the desired reference using the integrated search feature.
					\item Select the reference all the references you wish to import,as shown in \Cref{fig:JabRefWebSearchResults}.
					\item Click `Import entries' to import the selected entries.
					\item The references are added to your library.
				\end{enumerate}
				
				\begin{figure}[htbp]
					\centering
					\includegraphics[width=0.7\linewidth]{\insertimage{JabRef_Example_Web_Search.png}}
					\caption{Example Web Search Results for \texttt{``OSM-Classic''}.}
					\label{fig:JabRefWebSearchResults}
				\end{figure}
				
				\note{The Web Search tool by default uses a general search, however, a specific database can be chosen as well using the drop down arrow next to ``Search Selected''.}
				
			\subsubsection{Manual Entry}
				Manually adding a reference can be done in a `manual' and `automatic' way.
				When adding a new entry you will be faced with the option to select an entry type or to enter an ID (DOI, ArXiv, ISBN, \textit{etc.}).
				If you enter an ID, the information for the reference will be automatically pulled from the internet.
				Otherwise to manually enter all the information for a reference:

				\begin{enumerate}
					\item Click Library $\rightarrow$ `New entry' or use the shortcut `Ctrl + N' and this will show the following window.%
						\begin{center}%
							\includegraphics[width=0.5\linewidth]{\insertimage{JabRef_New_Entry.png}}
						\end{center}%
					\item Choose the entry type (\textit{e.g.}, article, book, inproceedings).
					\item Fill in the required fields like author, title, journal, \textit{etc}.
				\end{enumerate}
				
				By following these steps, you can efficiently populate your JabRef database with the necessary references.

		\subsection{Organizing References}
			Effectively organizing references is essential for a streamlined bibliography. 
			JabRef's grouping feature allows you to categorize references based on your preferences:
			\begin{enumerate}
				\item On the left panel, select `Add Group`.
				\item Give the group an appropriate name.
				\item Optionally you can add a Description, Icon, Colour, \textit{etc}.
			\end{enumerate}
			
			To assign a reference to a group:
			\begin{enumerate}
				\item Select the Reference(s) from the centre list.
				\item Drag them to the group on the left of the screen.
			\end{enumerate}
			\note{Groups can even be nested into groups to provide more levels of organization.}
			Organizing references into groups enhances accessibility and facilitates a more systematic approach to bibliography management.

	\section{Exploring Advanced Features of JabRef}
		JabRef's capabilities extend beyond the basics covered in the previous sections. 
		In this section, we'll explore some of the advanced features that enhance the efficiency and effectiveness of JabRef as a reference manager.

		\subsection{Quality Assurance: Checking and Correcting Entries}
			Ensuring the accuracy and completeness of references is crucial. 
			JabRef provides tools for quality assurance, allowing users to check and correct entries.

			To check for duplicate entries:
			\begin{enumerate}
				\item Click on `Quality $\rightarrow$ Find duplicates'.
				\item JabRef will identify and display duplicate entries.
			\end{enumerate}
			
			To correct entries:
			\begin{enumerate}
				\item Click on `Quality $\rightarrow$ Cleanup entries'.
				\item JabRef will provide some useful option to ensure conformity within the different references.
					This includes renaming Linked PDF's to match the standard of \texttt{``CitationKey - Title''}.
			\end{enumerate}
						
			These quality assurance features contribute to maintaining a clean and error-free bibliography.

		\subsection{Managing PDFs and File Links}
			JabRef facilitates the management of associated PDF's and file links, offering a consolidated approach to reference and document management.

			To link a PDF or file:
			\begin{enumerate}
				\item Open the entry editor for a reference.
				\item Click on `General' and use the `PDF' or `File' field to link the document.
			\end{enumerate}
			This integration helps to streamline the retrieval of PDF's or other associated documents directly from JabRef.
			Further, this allows JabRef to keep track of the comments and highlights in a single place (see \Cref{fig:JabRefFileAnnotations}).
			These annotations can be found by selecting the entry, and selecting the `File annotations' tab.

				\begin{figure}[htbp]
					\centering
					\includegraphics[width=0.7\linewidth]{\insertimage{JabRef_File_Annotations.png}}
					\caption{Showcase of the file annotations in JabRef.}
					\label{fig:JabRefFileAnnotations}
				\end{figure}

		\subsection{Additional Information}
			JabRef keeps track of a lot of information and can even help you with your research.
			Some additional information JabRef provides includes:
			\begin{itemize}
				\item Citation information
				\item Citation relationships (what the reference cites and who has cited the reference). 
					This further lets you open the links to the reference's source, or even add these references directly to your library.
				\item If one right clicks an entry you are provided the following options:
					\begin{itemize}
						\item Rank - rank the reference with one to five stars.
						\item Toggle Relevance - add a marker to show this is a relevant source.
						\item Priority - rank items as low, medium, or high priority.
						\item Read Status - set the status to read or skimmed.
					\end{itemize}
			\end{itemize}